\section{Projected Theorems, Lemma, and Corollary}
    \textbf{Theorem 1 (Correctness of Mapping).} \textit{For every activity in an input RDLT R, there exists a corresponding case in the set of output Petri Nets, and there does not exist a case from this PN that does not exist in the RDLT.} \\

    \noindent \textbf{Theorem 2 (Mapping of RBS in Petri Net).} \textit{Given an input RDLT R with an RBS, there exists a component in its corresponding output Petri Net that captures its reset and replenishment mechanism.} \\

    \noindent \textbf{Theorem 3 (Space complexity of mapping).} \textit{The space complexity of the proposed mapping is O(v), where v is the number of vertices in the RDLT R.} \\

    \noindent \textbf{Theorem 4 (Time complexity of mapping).} \textit{The time complexity of the proposed mapping is O($v^2$), where v is the number of vertices in the RDLT R.} \\

    \textbf{Lemma 1.} \textit{Given a pair of vertices x and y in the input RDLT R, if y is reachable from x, then there exists a firing sequence from $t_{1}$ to $t_{n}$ in the corresponding PN, where $t_{1} = t_{x}$ and $t_{n} = t_{y}$.} \\

    \textbf{Corollary 1 (Classical sound RDLT with an RBS).} \textit{Given an RDLT R that is Classical sound, there is a corresponding output PN that is Classical sound.} \\

    \textbf{Corollary 2 (Relaxed sound RDLT with an RBS).} \textit{Given an RDLT R that is Relaxed sound, there is a corresponding output PN that is Relaxed sound.}