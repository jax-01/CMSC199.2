\section{Statement of the Problem}
One way to decompose the case and process dimensions of RDLT is through the mapping of Petri Net. Recently, Sulla and Malinao \cite{sulla-malinao} have studied capturing the components of RDLT to Petri Net with emphasis on the $L$ and $M$-attributes of the former. The $L$-attribute of an arc represents the limit or the maximum number of times it can be traversed, while the $M$-attribute of a vertex signifies the center of a reset-bound subsystems (RBS) \cite{sulla-malinao}. However, further analyses are still needed to address numerous problems in the field of RDLT and Petri Net and its conversion from the former to the latter. For instance, it is still unclear if the activities of an input RDLT are consistent with the cases derived from the corresponding output Petri Net. Although results presented in \cite{sulla-malinao} establish relaxed and classical sound RDLT with relaxed or classical sound PNs, the full transformation of RBS and its reset capabilities are still not captured. Ultimately, the techniques used for mapping RDLT to Petri Net components presented in the literature do not fully capture the functionality of resets and replenishment in RBS. This implies that tokens are not replenished, and reuse is not allowed, unlike RBS. Additionally, the time and space complexity of completely mapping RDLT to Petri Net, with considerations of the full functionality of RBS, is not known.

% Do not add yet (for future works?) -> Furthermore, the existing literature for mapping RDLT to Petri Net does not tackle the possibility of simultaneous traversal of two different arcs, one from vertices inside an RBS, and another from a vertex inside the RBS to a vertex outside an RBS, which is an out-bridge. Currently, activity extraction is sequential.